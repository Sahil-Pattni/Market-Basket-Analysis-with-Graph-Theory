\documentclass[a4paper,11pt]{article}

% Padding Requirements
\usepackage[a4paper, margin=1in]{geometry}

	
% used for double spacing
\usepackage{setspace}
\doublespacing

% Used for bibliograpgy
\usepackage[style=ieee, sorting=none, backend=biber]{biblatex}
\addbibresource{references.bib}

\newcommand{\pcite}{\parencite}
% Do we need a table of contents? I don't think so.
\begin{document}
\title{Deliverable 1: Final Year Dissertation}
\author{
	Sahil Pattni\\ 
	BSc. Computer Science (Honours)\\\\
	Supervisor: Neamat El Gayar
	}
\date{\today}
\maketitle
\pagenumbering{arabic}
\newpage
\section*{Declaration}
I, Sahil Pattni, confirm that this work submitted for assessment is my own and is expressed in my own words. Any uses made within it of the works of other authors in any form (e.g., ideas, equations, figures, text, tables, programs) are properly acknowledged at any point of their use. A list of the references employed is included. 
\\
Signed: SMP\\
Date: \today


\section*{Abstract}
In this study, we will extract a minimum spanning tree (MST) from a data set using machine learning techniques. We will then use this minimum spanning tree to segment products together,  and produce association rules and extract the most interesting rules. The resulting rule-set will be compared to rules generated by the established Apriori Algorithm \pcite{apriori}, which is the foundation of association rule mining.  Additionally, we will study how the structure of our MST would change before and after promotional events, leading to insights that may help the management in such firms make better informed decisions about the type of promotions they would like to run.


\tableofcontents
\newpage

\section{Introduction}
\subsection{What is a Minimum Spanning Tree?}
Given an undirected graph $G$ with edges $E$ and vertices $V$ (i.e. $G(E,V)$), a \textit{spanning tree} can be described as a subgraph that is a tree \pcite{tree} which includes all the vertices $V$ of $G$ with the minimum number of edges required. A \textit{minimum spanning tree} is the spanning tree with the smallest sum of edge weights.
\subsection{Aims}
The aim of this study is to study the effectiveness of a minimum spanning tree in product classification and association rule mining,  in addition to determining how the architecture of the MST will change during and after large events such as promotions. The model will be tested on a relatively large dataset of sales data.
\subsection{Objectives}
The research objectives for this project have been laid out below, in the order that they will be carried out.
\begin{enumerate}
\item Acquire a suitable dataset upon which the MST can be constructed.
\item Investigate optimal MST generation algorithms (While Prim's \pcite{prims} \pcite{prims_og} and Kruskal's 
\end{enumerate}

\section{Background}

\section{Research Methodology}

\section{Evaluation Strategy}

\section{Project Management}


% Keep references on new page
\newpage
\printbibliography

% Keep appendix on new page
\newpage
\section*{Appendix}

\end{document}
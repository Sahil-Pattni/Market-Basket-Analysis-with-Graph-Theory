\documentclass[a4paper,11pt]{article}

% Padding Requirements
\usepackage[a4paper, margin=1in]{geometry}

	
% used for double spacing
\usepackage{setspace}
\doublespacing

% Used for bibliograpgy
\usepackage[style=ieee, sorting=none, backend=biber]{biblatex}
\addbibresource{references.bib}

% Do we need a table of contents? I don't think so.
\begin{document}
\title{Deliverable 1: Final Year Dissertation}
\author{
	Sahil Pattni\\ 
	BSc. Computer Science (Honours)\\\\
	Supervisor: Neamat El Gayar
	}
\date{\today}
\maketitle
\pagenumbering{arabic}
\newpage
\section*{Declaration}
I, Sahil Pattni, confirm that this work submitted for assessment is my own and is expressed in my own words. Any uses made within it of the works of other authors in any form (e.g., ideas, equations, figures, text, tables, programs) are properly acknowledged at any point of their use. A list of the references employed is included. 
\\
Signed: Sahil Manojkumar Pattni\\
Date: \today


\section*{Abstract}
In this digital age, data is being generated at an exponential rate, with data analytics being used by corporations and small businesses alike to produce insights, reduce costs, optimize operations and increase profits.  Generating association rules allow us to find non-intuitive associations that bring new insights to the management, allowing them to leverage this information to maximize their profits. A prime example would be the \textit{'Beers and Diapers'} case \cite{beers_diapers}, where a company looked at their point-of-sale data and found a strong association between beers and diapers being co-purchased, which seems rather unintuitive.
\\In this study, we will extract a minimum spanning tree (MST) from a data set using machine learning techniques. We will then use this minimum spanning tree to segment products together,  and produce association rules and extract the most interesting rules. The resulting rule-set will be compared to rules generated by the established Apriori Algorithm \cite{apriori}, which is the foundation of association rule mining.  Additionally, we will study how the structure of our MST would change before and after promotional events, leading to insights that may help the management in such firms make better informed decisions about the type of promotions they would like to run.

\newpage
\tableofcontents
\newpage

\section{Introduction}
\subsection{What is a Minimum Spanning Tree?}
Given an undirected graph $G$ with edges $E$ and vertices $V$ (i.e. $G(E,V)$), a \textit{spanning tree} can be described as a subgraph that is a tree \cite{tree} which includes all the vertices $V$ of $G$ with the minimum number of edges required. A \textit{minimum spanning tree} is the spanning tree with the smallest sum of edge weights.

\subsection{Context}
For business which deal with the sale of a heterogeneous physical assets - such as groceries, hypermarkets and select retail outlets - operations such as inventory management and product placement play an instrumental role in determining the business' financial success. These involve asking questions such as:
\begin{itemize}
\item Which products should be placed at the entrance of the store? Which should be placed closer to the exit?
\item Which products will benefit the most by being placed at eye-level?
\item Which products should be placed next to each other to maximize the purchase volume?
\end{itemize}
One way to find optimal solutions for such questions is to employ the use of Association Rule Mining techniques. These techniques 
Several algorithms and techniques exist for association rule mining, such as the Apriori Algorithm \cite{apriori} and FP-Growth \cite{fp_growth}.

\subsection{Aims}
The aim of this study is to study the effectiveness of a minimum spanning tree in product classification and association rule mining,  in addition to determining how the architecture of the MST will change during and after large events such as promotions. The model will be tested on a relatively large dataset of sales data.

\subsection{Objectives}
The research objectives for this project have been laid out below, in the order that they will be carried out.
\begin{enumerate}
\item Acquire a suitable dataset upon which the MST can be constructed.

\item \label{optimal} Explore and evaluate MST extraction algorithms.
\\(While Prim's \cite{prims}\cite{prims_og} and Kruskal's \cite{kruskal} are the most commonly used algorithms to extract the MST from a graph, research has been conducted on more efficient ways to extract the MST using machine learning techniques such as Artificial Neural Networks \cite{ann_mst} and K-Nearest-Neighbors \cite{knn_mst}).

\item Generate an undirected graph $G(E,V)$ where the edges $E$ are the correlation values between the product vertices.

\item Extract a minimum spanning tree from this graph using the technique determined in Step \ref{optimal}. 

\item Analyze this MST and use its architecture to determine product clusters and generate association rules.

\item Generate association rules 

\end{enumerate}

\section{Background}

\section{Research Methodology}

\section{Evaluation Strategy}

\section{Project Management}


% Keep references on new page
\newpage
\printbibliography

\end{document}
\chapter{Introduction}


\section{Motivation}
For businesses such as groceries, hypermarkets, and retail outlets that deal with the trade of heterogeneous physical assets, operations such as inventory management and product placement play an instrumental role in determining the business' financial success. These involve asking questions such as:
\begin{itemize}
\item Which products should be placed at the entrance of the store? Which should be placed closer to the exit?
\item Which products will benefit the most by being placed at eye-level?
\item Which products should be placed next to each other to maximize the purchase volume?
\end{itemize}
One way to find optimal solutions to such queries is to employ the use of Association Rule Mining (also known as Market Basket Analysis). This set of techniques assess frequent itemsets (e.g. from sales data) and generate association rules between products. 
A prime example of the utility of association rules is the urban legend of \textit{"Beers and Pampers"}, where a company allegedly studied their point-of-sale data and found a strong association between the purchase of diapers and a particular brand of beer during a certain time. With \textit{diapers} as the antecedent and \textit{beer} as the consequent, this rule can be written as:
\[
\{\textit{Diapers}\} \rightarrow \{\textit{Beer}\}
\]
This is an example of a single-element rule, where both the antecedent and consequent are sets that contain only one element each. Several algorithms exist for association rule mining, most prominently the Apriori Algorithm \cite{apriori} and FP-Growth \cite{fp_growth}, however these algorithms tend to generate an overwhelming amount of rules, rendering it inconvenient for the end-user to extract actionable information from the ruleset. This paper proposes to improve upon an existing method (see: Section \ref{sec:mst paper})  that derives association rules from minimum spanning trees.

\section{Aims}
The aim of this paper is to improve upon the aforementioned method by introducing a method that allows for the generation of multi-element association rules (i.e. rules where either/both the antecedent and consequent of the rule contain more than one element).

\section{Objectives}
The research objectives for this paper are as described below:
\begin{enumerate}
\item Acquire a suitable dataset upon which the study can be conducted.
\item Construct an affinity graph from the chosen dataset.
\item Explore and evaluate MST extraction algorithms.
\item Identify and evaluate methodologies for generating multi-element association rules from an MST.
\item Extract MST from the graph and generate association-rules using the chosen methodologies.
\item Evaluate the generated association rules against the rules generated by the Apriori Algorithm.
\end{enumerate}
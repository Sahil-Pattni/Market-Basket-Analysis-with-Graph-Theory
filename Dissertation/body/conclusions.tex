\chapter{Conclusions}
\section{Achievements}
In this paper, we have introduced a new algorithm for association rule generation - the \algo\ algorithm: an extension of the methodology proposed by \pcite{mst_paper} that also employs the techniques and principles outlined in \pcite{apriori} and \pcite{markov_clustering}. The \algo\ algorithm produces bi-cluster rules where the antecedent and consequent are from separate clusters, and intra-cluster rules where both the antecedent and consequent originate from the same cluster. In doing so, it appears to capture the high-support rules from all possible rules within any support and confidence constraints.  We have also theorized that association rules that are bi-cluster or intra-cluster tend to have a higher support score than those rules that are not. The rules generated by the \algo\ algorithm seem to also follow general intuition (see: Table \ref{tab:arm_rules}) such as \texttt{\{cigarettes\}} $\rightarrow$ \texttt{\{chewing gum and candy\}}, and \texttt{\{filters\}} $\rightarrow$ \texttt{\{lubricant}\}, further reinforced by their existence in the rules generated by the Apriori algorithm as well.

\section{Limitations}
\textbf{Dataset Bias}\\
All analysis was conducted on the topological structures generated from a singular dataset. This may not be wholly representative of the rules the \algo\ algorithm may generate, and perhaps a different dataset may have yielded different results.
\\\textbf{Clustering}\\
As seen in Figure \ref{fig:cluster_named} on Page \pageref{fig:cluster_named}, the Markov Clustering configuration yielded some clusters that seem appropriate, yet some (for example, the largest) were perhaps too broad to classify under a single cluster.
\\\textbf{Rule Structure}\\
The \algo\ algorithm's key feature is both its greatest strength and drawback. Due to the fact that our algorithm only selects those rules where the items in a given itemset are from the same cluster, the algorithm overlooks rules where the antecedent and consequent have a composition of items from multiple clusters. 
Figures \ref{fig:rule_support} and \ref{fig:rule_lift} indicate that the \algo\ algorithm did not generate several higher-value rules present at the elbow of the Apriori rule curve.
However, we have also concluded via Figure \ref{fig:rule_support} that this constraint captures a majority of high-value (i.e. high support) rules and does not capture the lower-value rules within the support and confidence constraints.


\section{Future Works}
As mentioned, a key limitation of our analysis is that it only applies to the dataset we used. Future projects involving different datasets may produce new insights into the advantages and/or limitations of the algorithm. Additionally, alternative clustering algorithms may be considered as a replacement for the Markov Clustering algorithm. 
